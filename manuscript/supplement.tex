\documentclass[a4paper,man,natbib]{apa6}

\usepackage[english]{babel}
\usepackage[utf8x]{inputenc}
\usepackage{amsmath}
\usepackage{graphicx}
\usepackage[colorinlistoftodos]{todonotes}

% override apa6 first paragraph index
% https://tex.stackexchange.com/questions/155028/indentation-apa6-class-for-2nd-paragraphs-after-title
% \makeatletter
%   \b@level@one@skip=-2.5ex plus -1ex minus -.2ex
%   \b@level@two@skip=-2.5ex plus -1ex minus -.2ex
% \makeatother

\title{Supplemental materials for ``An item response theory analysis of the Matrix Reasoning Item Bank (MaRs-IB)''}
\shorttitle{Supplemental materials for ``IRT analysis of MaRs-IB''}
\author{Samuel Zorowitz$^1$, Nathaniel D. Daw$^{1,2}$}
\affiliation{$^1$Princeton Neuroscience Institute, Princeton University, USA\\$^2$Department of Psychology, Princeton University, USA}

\setcounter{figure}{0}
\setcounter{table}{0}
\renewcommand{\thetable}{S\arabic{table}}
\renewcommand{\thefigure}{S\arabic{figure}}

\begin{document}
\maketitle

\section*{Speed-accuracy trade-offs in Chierchia et al. (2019)}

To investigate the possibility of speed-accuracy trade-offs in the original MaRs-IB data, we looked at the proportion of correct responses to the easiest items (dimension 1/2 items) as a function of the number of participants having reached that item. In all partitions of these data, we found robust positive correlations between proportion correct and number reached (dimension 1 items: $\rho$ = 0.514, p = 0.050; dimension 2 items: $\rho$ = 0.767, p < 0.001; combined: $\rho$ = 0.695, p < 0.001). This result suggests that the participants that did reach items later in the sequence were able to do by sacrificing accuracy for speed. As such, the item-level performance summary released as part of Chierchia et al. (2019) are likely biased indicators of item functioning.

\begin{table}
\centering
\begin{tabular*}{\textwidth}{ccccccllll}
\toprule
 & & & & & & \multicolumn{4}{c}{Spearman rank correlation} \\
\cmidrule(lr){7-10}
Measure & & Mean (SD) & & IQR & &  NFC-10 & PCF-8a & SNS & MaRs-SF \\
\midrule
NFC-10 & & 25.03 (8.27) & & 20.00 -- 31.00 & & - &  &  & \\
PCF-8a   &  & 22.17 (6.29) & & 18.75 -- 26.25 & & 0.27** &  - &  &  \\
SNS   &  & 29.07 (7.48) & & 25.00 -- 35.00 & & 0.46** &  0.29** &  - &   \\
MaRs-SF & &   8.00 (2.53) & &   6.00 -- 10.00 & & -0.04 &  0.04 &  0.14* &  - \\
\bottomrule
\end{tabular*}
\captionsetup{width=1.\textwidth}
\caption{\normalfont Correlations between performance on the MaRs-IB short forms and self-report measures. IQR = interquartile range; NFC-10 = need for cognition (10-item) scale; PCF = PROMIS cognitive functioning scale (8a); SNS = subjective numeracy scale. \\ ** p < 0.001,  * p < 0.05 (not corrected for multiple comparisons)}
\end{table}
\end{document}